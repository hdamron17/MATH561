\documentclass{homework}

\title{Problem Set 1}
\author{Hunter Damron, Jacob Folks, Syd Miyasaki, Jeffrey Russell}
\date{September 12, 2018}

\begin{document}
	\maketitle
	\begin{enumerate}
		\setcounter{enumi}{10}
		\item Using $(\N; +, \cdot)$
		\begin{enumerate}
			\item $\{0\} \coloneqq \forall x \left[x + y = x = y + x\right]$.
			\item $\{1\} \coloneqq \forall x \left[x \cdot y = x = y \cdot x \right]$.
			\item $\{\langle m,n \rangle \mid n \text{ is the successor of } m \text{ in } \N\} \coloneqq n = m + 1$.
			\item $\{\langle m,n \rangle \mid m < n \text{ in } \N\} \coloneqq \exists x \left[m + x = n\right]$.
		\end{enumerate}
		\item Using $\mathfrak{R} = (\R; +, \cdot)$
		\begin{enumerate}
			\item $[0, \infty) \coloneqq \exists x \left[x \cdot x = y\right]$.
			\item $\{2\} \coloneqq \forall x [x \cdot y = x + x]$.
			\item That any finite union of intervals, the endpoints of which are algebraic, is definable on $\mathfrak{R}$.
			\begin{proof}
				\TODO.
			\end{proof}
		\end{enumerate}
		
		\setcounter{enumi}{13}
		\item What subsets of $\R$ are definable in $(\R, <)$? \TODO
		
		What subsets of $\R\times\R$ are definable in $(\R, <)$? \TODO
		
		\setcounter{enumi}{15}
		\item Give a sentence having models of size $2n$ for every positive integer $n$ but no finite models of odd size. (Here the language should include equality and will have whatever parameters you choose.) \emph{Suggestion}: One method is to make a sentence that says, ``Everything is either red or blue, and $f$ is a color-reversing permutation.''
		
		\TODO
	\end{enumerate}
\end{document}